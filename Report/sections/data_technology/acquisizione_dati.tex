Nel dataset \textsc{WNV Mosquito} abbiamo voluto aggiungere l'attributo 
\texttt{day\_of\_week} che corrisponde al giorno della settimana in cui è stato 
effettuato il test. Questo è stato utile per derivare delle statistiche, 
come per esempio il giorno della settimana in cui è stato effettuato il maggior 
numero di test, nonché come feature da fornire ai modelli addestrati.
\\

Inoltre, è stato creato un nuovo dataset contenente esclusivamente le colonne 
\textsc{Block}, \textsc{Latitude} e \textsc{Longitude} del dataset \textbf{WNV 
Mosquito}. La tabella risultante è stata denominata \textsc{Block} e verrà 
utilizzata durante il processo di integrazione dei dati.
\\

Un analisi del dataset \textsc{Weather} ha poi mostrato come alcuni codici 
della colonna \texttt{code\_sum} comparissero molto raramente. Per rendere 
questi valori più adatti all'utilizzo come feature, abbiamo deciso di 
aggregarli manualmente in categorie più ampie, in base alla somiglianza dei 
fenomeni meteorologici rappresentati:
\begin{itemize}
	\item I codici \texttt{GR} (grandine, 1 occorrenza), \texttt{GS} (piccola 
	grandine, 3 occorrenze) e \texttt{PL} (pioggia gelata, 33 occorrenze) sono 
	stati trasformati nel codice aggregato \texttt{HAIL} (grandine, 37 
	occorrenze).
	
	\item I codici \texttt{SQ} (raffiche di vento, 8 occorrenze), \texttt{FC} 
	(nubi a imbuto, 1 occorrenza) sono stati trasformati nel codice aggregato 
	\texttt{WIND} (forte vento, 9 occorrenze).
	
	\item I codici \texttt{FU} (fumo, 19 occorrenze), \texttt{DU} (polvere, 1 
	occorrenza) e sono stati trasformati nel codice aggregato \texttt{SMOKE} 
	(fumo e polvere, 20 occorrenze).
	
	\item Tutti i rimanenti codici compaiono più di 110 volte nell'arco 
	temporale analizzato e sono stati mantenuti invariati.
\end{itemize}

Un'ulteriore operazione effettuata su \textsc{Weather} è stata aggregare le 
rilevazioni giornaliere originali in rilevazioni settimanali, in preparazione 
della successiva fase di Data Fusion. Questo è stato necessario per allineare 
il dataset a \textsc{WNV Mosquito}, nel quale i test seguono una cadenza 
settimanale. A partire dai dati meteo giornalieri sono state calcolate una 
serie di colonne aggregate, usando diverse strategie:
\begin{itemize}
	\item La colonna \texttt{year\_month\_day} è stata aggregata in 
	\texttt{year} e \texttt{week\_of\_year}.
	
	\item È stata aggiunta la colonna \texttt{days\_per\_week}, rappresentante 
	il numero di giorni utilizzati nell'aggregazione. Questo perché i giorni a 
	cavallo tra un anno ed il successivo vengono aggregati in due record 
	distinti (ad esempio settimana 53 del 2017 e settimana 1 del 2018). Questa 
	colonna ci ha permesso di verificare che nel nostro periodo di interesse 
	(maggio--ottobre) l'aggregazione fosse sempre completa.
	
	\item Per tutte le colonne numeriche è stato calcolato il valore medio 
	settimanale.
	
	\item Delle colonne \texttt{t\_max} e \texttt{t\_min} sono stati calcolati 
	rispettivamente il massimo e il minimo settimanale.
	
	\item Le colonne booleane \texttt{*\_is\_suspicious} e 
	\texttt{code\_sum\_*} sono state trasformate in colonne numeriche 
	contenenti il numero di valori \texttt{true} di quella settimana.
\end{itemize}
