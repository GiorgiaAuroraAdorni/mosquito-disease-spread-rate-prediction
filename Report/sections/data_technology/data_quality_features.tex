Dopo le operazioni di integrazione dei quattro dataset, tre iniziali e uno 
generato, sono state selezionate 70 features.

In particolare sono stati eliminati dal dataset \textsc{Stations} i record 
relativi alle stazioni con identificativo \texttt{wban} {4807} e {4879}, poiché 
le rilevazioni meteo disponibili, relative a queste stazioni, hanno raccolto 
dati solo a partire dal 2015, lasciando scoperti i primi 8 anni in cui sono 
stati effettuati i test del WNV.

Per quanto riguarda il dataset \textsc{Weather}, sono stati eliminati gli 
attributi \texttt{snow\_water}, \texttt{snow\_depth}, \texttt{depart}, 
\texttt{sunrise}, \texttt{sunset}, \texttt{snow\_fall} e le relative colonne 
\texttt{flag}, poiché per le stazioni in esame non sono stati misurati questi 
parametri.

Le colonne del dataset \textsc{Block} non sono state utilizzate all'interno del 
database integrato, poiché duplicate nella tabella \textsc{WNV Mosquito}.\\
\\
Dopo l'integrazione, il database risultante contiene 25482 righe e 70 colonne, 
in particolare, analizzando dataset per dataset si hanno:
\begin{itemize}
	\item \textsc{WNV Mosquito}: contenente 27196 righe e 12 attributi;
	\item \textsc{Weather}: contenente 1701 righe e 50 attributi;
	\item \textsc{Station}: contenente 3 righe e 8 attributi.
\end{itemize}

%Di questi attributi ... potrebbero influire sulla predizione dell'algoritmo di 
%Machine Learning.

\section{Completezza}
In seguito alla selezione delle 70 features, la completezza complessiva del 
dataset risulta del 100\%, così come quella di attributo e di tupla.

\section{Leggibilità}
