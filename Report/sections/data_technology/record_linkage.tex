Tra i record delle tabelle \textsc{WNV Mosquito} e \textsc{Weather} vi è una 
relazione di tipo spazio-temporale che vorremmo sfruttare per creare le 
features su cui addestrare i modelli di machine learning. Vorremmo in 
particolare collegare ogni esito del test di positività al WNV alle condizioni 
meteorologiche della settimana in cui è stato effettuato, rilevate dalla 
stazione meteorologica più vicina a dove sono state catturate le zanzare.

Per questo motivo abbiamo applicato un processo di record linkage e data fusion 
che ci ha portato a costruire un unico dataset fuso a partire dalle due 
tabelle. Innanzitutto, tramite reverse-engineering dei dati abbiamo steso lo 
schema concettuale riportato in Figura~\ref{fig:er-schema}. 

\begin{figure}[H]
    \centering
    \def\svgwidth{\columnwidth}
    \scalebox{.5}{
        \input{images/ER.pdf_tex}
    }
    \caption{Schema concettuale per il record linkage}
    \label{fig:er-schema}
\end{figure}

In particolare, sono state definite le seguenti chiavi primarie per le entità:

\begin{itemize}
    \item \textsc{WNV Mosquito}: (\texttt{test\_id}).

    \item \textsc{Block}: (\texttt{block}).
    
    \item \textsc{Station}: (\texttt{wban}).
    
    \item \textsc{Weather}: (\texttt{wban}, \texttt{year}, 
    \texttt{week\_of\_year}).
\end{itemize}

Le relazioni tra entità invece si basano sulle seguenti chiavi e in particolare 
per la relazione \textbf{NearestNeighbour} sul concetto di vicinanza fra 
elementi distribuiti nello spazio.

\begin{itemize}
    \item \textbf{Trap}: \textsc{WNV Mosquito}(\texttt{block}), 
    \textsc{Block}(\texttt{block}).

    \item \textbf{Observation}:  \textsc{Weather}(\texttt{wban}), 
    \textsc{Station}(\texttt{wban}).
    
    \item \textbf{NearestNeighbour}: il concetto di vicinanza spaziale è 
    ottenuto effettuando il prodotto cartesiano per tutte le coppie
    \textsc{Block}$\times$\textsc{Station} e calcolando la distanza tra le 
    coordinate geografiche ottenute dagli attributi \texttt{latitude} e 
    \texttt{longitude} sia di \textsc{Block} che di \textsc{Station},
    coinvolgendo nella relazione solamente la \textsc{Station} più vicina ad 
    un \textsc{Block}.
\end{itemize}

Poiché le chiavi scelte per le relazioni non presentano errori né ambiguità, 
non è stato necessario definire una distanza di edit per effettuare il record 
linkage: è stato sufficiente cercare corrispondenze esatte tra le rispettive 
chiavi.
