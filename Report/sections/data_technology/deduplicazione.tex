Una volta terminata la valutazione dei dateset tramite le dimensioni qualità, è stata eseguita una deduplicazione dei dataset, ovvero l'identificazione di tutte le coppie o gruppi di tuple corrispondenti allo stesso oggetto del mondo reale.

La deduplicazione è stata eseguita utilizzando il software Power BI in particolare ricorrendo allo strumento Power Query per l'inserimento, la trasformazione, l'integrazione e l'arricchimento dei dati.
\\\\	
In particolare è stata effettuata la deduplicazione del dataset \textbf{WNV Mosquito}, utilizzando come attributi \textsc{Block}, \textsc{Latitude} e \textsc{Longitude}. 
Su questo dataset è stata verificata la presenza di alcune tuple duplicate, che differivano per i campi \textsc{Latitude} e \textsc{Longitude}. Calcolando la distanza tra le diverse coordinate abbiamo notato che le differenze erano solo di alcuni metri.
Abbiamo dunque deciso di tenere randomicamente una delle tuple duplicate data la scarsa differenza tra le due e la poca importanza della precisione, poichè nel nostro modello predittivo le coordinate di ogni indirizzo vengono utilizzate solo per cercare la stazione di rilevamento meteo più vicina.

