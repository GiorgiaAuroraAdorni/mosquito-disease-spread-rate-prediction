La leggibilità è una dimensione di qualità utile per permettere, a chi vuole 
utilizzare una collezione di dati, di analizzare almeno sommariamente le 
informazioni in modo da poter prendere determinate decisioni riguardanti 
l'utilizzo dei dati, come ad esempio quali informazioni potrebbero essere più 
significative per un modello predittivo di machine learning.

I datasets scelti per il progetto utilizzano unità di misura del 
sistema anglosassone, dunque di difficile lettura per chi utilizza il sistema 
internazionale, oppure vengono impiegate sigle alfanumeriche per indicare 
un determinato tipo di valori. 

Di seguito verranno discusse le modifiche 
apportate agli attributi dei vari datasets per migliorarne la leggibilità.

\begin{itemize}

    \item \textbf{Weather}: Il dataset utilizza il valore \textit{M} per 
        indicare un valore mancante per un attributo, in questo caso è stato 
        scelto di sostituire a \textit{M} il valore \textit{null} poichè viene 
        riconosciuto correttamente come valore mancante dai software per 
        l'import/export dei dati e dai DBMS.
        
        \begin{itemize}
            
            \item \texttt{Date}: inizialmente l'attributo rappresentava una 
                data attraverso il pattern \textit{yyyyMMdd}, per migliorarne 
                la leggibilità è stato cambiato il pattern e utilizzato nella 
                forma \textit{dd/MM/yyyy}. Ad esempio la data 
                \textit{20070501} è stata modificata in \textit{01/05/2007}.
        
            \item \texttt{Tmax}, \texttt{Tmin}, \texttt{DewPoint}, 
                \texttt{WetBulb}, \texttt{Heat}, \texttt{Cool}: i valori di 
                temperatura sono stati convertiti da gradi Fahrenheit (F) 
                a gradi Celsius (C).

            \item \texttt{StnPressure}, \texttt{SeaLevel}: i valori di 
                pressione sono stati convertiti da pollici di mercurio (inch) a 
                millimetri di mercurio (mm).

            \item \texttt{ResultSpeed}, \texttt{AvgSpeed}, \texttt{Max5Speed}, 
                \texttt{Max2Speed}: i valori di velocità sono stati convertiti 
                da miglia orarie (mph) a chilometri orari (kmh).
                
            \item \texttt{CodeSum} %elia

        \end{itemize}
    
\end{itemize}


\begin{itemize}
	
	\item \textbf{WNV mosquito}:
	
	\begin{itemize}
		
		\item \texttt{Trap}: nel dataset le trappole vengono contrassegnate dalla lettera \textit{T} seguita da tre cifre. Le trappole di tipo "satellitare" sono installate di solito vicino alle trappole principali e servono per migliorare gli sforzi di sorveglianza. Per questo motivo esse sono indicate dallo stesso pattern della trappola principale con in aggiunta una lettera. Per esempio T220A è la prima trappola satellitare per T220. Essendo questo di difficile lettura sono state aggiunte due colonne: \texttt{Main Trap} che riporta il codice della trappola principale e \texttt{Sub Trap} che contiene nel caso delle trappole satellite la lettera ad essa associata.
		
		\item \texttt{Result}: inizialmente la colonna conteneva dei valori di testo \textit{positive} e \textit{negative}, quesi sono stati trasformati in valori logici booleani.  
		
		\item \texttt{Number Of Mosquito}: nel caso di numero di zanzare superiore a 50, questo viene diviso in un altro record con gli stessi attributi in modo tale che il numero di zanzare sia limitato a 50.  
			
	\end{itemize}
	
\end{itemize}

