\section{Descrizione del dominio di riferimento}

Il virus del Nilo occidentale è più comunemente diffuso agli esseri umani attraverso le zanzare infette. Circa il 20\% delle persone che si infettano con il virus sviluppa sintomi che vanno da una febbre persistente a gravi malattie neurologiche che possono portare alla morte.

Nel 2002, i primi casi umani di virus del Nilo occidentale furono riportati a Chicago, così che due anni dopo il Dipartimento della sanità pubblica (CDPH) stabilì un programma completo di sorveglianza e controllo, tuttora in vigore.

Ogni settimana, dalla fine della primavera fino all'autunno, le zanzare vengono catturate nelle trappole della città e viene testata la presenza del virus. 
%I risultati di questi test influenzano quando e dove la città spruzzerà i pesticidi dispersi nell'aria per controllare le popolazioni di zanzare adulte.

Grazi ai dati relativi alle condizioni meteorologiche e all'ubicazione delle trappole il nostro obiettivo è quello di prevedere se le zanzare risultano positiva al virus del Nilo occidentale.

%quando e dove diverse specie di zanzare saranno testate positive per il virus del Nilo occidentale. 

%Un metodo più accurato di previsione dei focolai del virus del Nilo occidentale nelle zanzare aiuterà la città di Chicago e il CPHD ad allocare in modo più efficiente ed efficace le risorse per prevenire la trasmissione di questo virus potenzialmente letale.

% L’obiettivo di questo elaborato è ... sfruttando le moderne tecniche di machine learning.


\section{Scelte di design per la creazione del dataset}

\section{Eventuali ipotesi o assunzioni}

