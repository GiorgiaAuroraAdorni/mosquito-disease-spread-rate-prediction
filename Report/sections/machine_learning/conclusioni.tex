Gli esperimenti effettuati hanno mostrato un similarità di performance 
per tutti i modelli implementati. In particolare possiamo notare come in 
nessun modello il valore medio di Area Under Curve (AUC) sia inferiore a 
$0,80$, che generalmente viene considerato come un indice di buone prestazioni. 
I modelli predicono correttamente i valori negativi e in riferimento 
agli obbiettivi prefissati per il progetto questo risulta essere un vantaggio 
poichè ridurrebbe considerevolmente il costo per le analisi di laboratorio.

\section{Sviluppi futuri}

Per quanto riguarda \textbf{Random Forest}, poiché l'algoritmo offre la 
possibilità di misurare l’importanza della variabile predittore, uno sviluppo 
futuro potrebbe essere utilizzare l'importanza delle variabili, mostrata nella 
figura \ref{fig:rf_importance}, per classificare l'utilità delle variabili, ed 
utilizzare solo le più importanti come feature del modello. 

\begin{figure}[H]
	\centering
	\includegraphics[width=0.6\columnwidth]{images/ml/random_forest/HoldoutRF/model_importance}
	\caption{Random Forest Importance}
	\label{fig:rf_importance}
\end{figure}

